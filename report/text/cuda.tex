
Improving the performance of our algorithms can be done in many ways,
but one of the more obvious ones are using parallel computing!

Currently, the most accessible way to run programs in parallel is
using the graphics computation unit (GPU). Modern GPUs are very
powerful, and major manufactures have released software development
kits (SDK) for utilise the GPU for general purpose computation.

One of these SDKs is nVidias CUDA\cit{cuda}. The only thing needed to
use CUDA is a nvidia graphics card that is relatively new (a few years
tops), and the free SDK found at the CUDA website.

As GPUs were developed to render graphics, they are optimized to work
on spatially coherent data. This makes many of our algorithms a
natural target, as we often only need information about neighbouring
data points.

\todoPtx{CUDA}

%%% Local Variables: 
%%% mode: latex
%%% mode: auto-fill
%%% TeX-PDF-mode: t
%%% TeX-master: "../master.tex"
%%% End: 
