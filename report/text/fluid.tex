We have previously discussed the advantages of using level set methods
to evolve interfaces by representing them implicitly rather than
explicitly. One area in which this proves particularly useful is in
the simulation of physical phenomena. We look at one such example in
this section: the simulation of fluids.  The basic idea is to let the
fluid be represented by the zero-level isosurface of a signed distance
field. By solving a set of equations we arrive at a velocity field,
with which the interface can be evolved as discussed in the section on
generating motion by externally generated velocity fields, section
\vref{sec:extVel}.

\section{Fluid equations}
The basis of our fluid simulator is the \emph{incompressible
  Navier-Stokes equations} as presented in \cit{bridson2008fluid}, a
common approach to fluid simulation. These are a set of equations
which describe the motion of an incompressible
fluid. Incompressibility is a central quality of fluids. Simply put,
it means that the fluid cannot grow or shrink in size. Naturally this
property does not quite hold for real-world fluids, especially if we
include gases in the definition, as fluids in the real world do
compress to some degree. For graphical simulation purposes, however,
it is adequate to assume complete incompressibility.


Here, then, is how the incompressible Navier-Stokes equations are
usually written:
\begin{align}
  \label{eq:momentum}
  \frac{\partial\vec{u}}{\partial t}+\vec{u} \cdot \nabla\vec{u} + \frac{1}{\rho}\Delta p &= \vec{g}+\nu\nabla\cdot\nabla\vec{u},\\
  \label{eq:incomp}
  \nabla\cdot\vec{u}&=0
\end{align}
We will be breaking these equations down into some simpler parts that
can be calculated discretely and numerically without too much
hassle. For now, the variables of the equations: $\vec u$ is the field
of velocities that affect the fluid, $\rho$ is the density of the
fluid, $p$ tracks how much pressure is present across the fluid,
$\vec{g}$ is the downwards force of gravity acting upon the fluid and
$\nu$ is a term known as viscosity, which basically says something
about how easily the fluid deforms.

Equation \vref{eq:momentum} is called the \emph{momentum equation} and
deals (not surprisingly) with the momentum of the fluid. That is, how
the fluid continues to move once in motion. Naturally, this equation
also encompasses motion derived from external forces like gravity.

The incompressibility of the fluid is captured by equation
\vref{eq:incomp}. This equation is derived from looking at the normal
component of the velocity at the boundary of a given fluid. In
basic terms, it says that the amount of fluid flowing out of the fluid
region is equal to the amount flowing in. This measure of change is
called divergence, and by this restriction we seek to remove it with
as much numerical accuracy as possible. When evolving the interface
with the resulting velocity field, the incompressibility constraint
ensures that the interface is never moved in a way that produces a
larger or smaller fluid volume (or area, in the case of our two
dimensional implementation).

Looking at the equations, solving the entire scheme at once seems
daunting. Thankfully, \cit{bridson2008fluid} presents us with a method
for breaking them into smaller, more managable parts. By using this
technique, called splitting, we arrive at the following parts:
\begin{align}
  \label{eq:advection}
  \frac{Dq}{Dt}&=0 \text{    (advection),}\\
  \label{eq:forces}
  \frac{\partial \vec{u}}{\partial t} &= \vec{g} \text{    (body forces),}\\
  \label{eq:pressure}
  \frac{\partial \vec{u}}{\partial t} + \frac{1}{\rho}\nabla p &= 0\\
  \label{eq:divergence}
  \text{such that } \nabla\cdot\vec{u}&=0 \text{    (pressure/incompressibility),}
\end{align}

Notice that the momentum equation has been split into three seperate
parts. The first, advection, ensures that velocities are passed around
the velocity field properly. The second simply applies external
forces. The third and fourth adjust the velocity field under the
incompressibility constraint, and thus the velocity field is created
and maintained across steps. To formalize the preceding into an
algorithm:
\begin{itemize}
\item Advect velocities around the velocity field
\item Add external forces
\item Adjust the velocity field for pressure
\item Use the resulting velocity field to move the fluid
\end{itemize}
As mentioned, we use a signed distance function to represent the
position and layout of the fluid. The zero-level isocontour is the
surface of the fluid and naturally divides the domain into two
sections: all values above zero represent empty space not currently
occupied by the fluid while everything below zero represents the
interior of the fluid.

Our natural inclination at this point would be to represent the
discrete velocity field as an identically sized double array (or triple,
depending on dimensionality) of values. Each cell would hold the
velocity of the fluid at that point, which seems the simplest way of
storing it. However, to ease certain portions of later sections of the
simulation we have chosen instead the "Marker and Cell" (or MAC for
short) grid which \cit{bridson2008fluid} credits to Harlow and Welch.

\section{Marker and cell grid}
For a two-dimensional fluid domain, the velocities clearly need to be
two-dimensional as well. With the MAC grid we chose not to simply
store the complete velocity vector of the fluid at the center of each
cell of the fluid. Instead, we store the only the normal components of
the velocity at the \textit{edges} of cells.\image{mac.png}{0.3}{The
  MAC grid stores velocities at the edge of cells, rather than the
  center}{fig:mac} As figure \vref{fig:mac} illustrates, each edge of
cell in the signed distance field has associated with it a value
indicating the component of the velocity vector in the corresponding
direction. In practical terms, we store two grids in addition to the
discretized signed distance field, one for each spatial dimension of
the velocity field.

The purpose of this initially unintuitive representation becomes
apparent when need to calculate central differences in the velocity
field. With a "traditional" velocity field, computing the signed
distance field we would sample velocities at either side of the cell
in question, ignoring the value stored at the cell itself. Since our
field may contain very thin feature-variation, this method of central
differencing may accidentally ignore "sharp" features of no more than
a cell's width.

With the MAC grid, when central differencing we can sample the
"half-values" at the edges of cells, thereby efficiently avoiding this
problem at no additional cost.

The MAC grid also needs to be able to handle values being taken
between grid points. Not only at the cell edges, but at arbitrary
locations in the grid, particularly at grid centers. We will see why
in this next section.

\section{Advection and forces}
With the grids thus in place, we may begin our algorithm. Referring to
the overview, the first step is the advection of the velocity
field. In this step we allow motion to carry on, retaining momentum as
the fluid moves. The implementation of this step is quite intuitive.

Recall equation \vref{eq:advection} which describes the advection
step. With our grid-based representation of the fluid it seems obvious
to use an Eulerian solution to the equation. We won't go into that,
however, since it proves to be conditionally unstable. Instead we use
something called the \textit{semi-Lagrangian} method. We have
previously discussed the difference between Eulerian and Lagrangian
viewpoints. As the name suggests, this method is a mix of the two,
taking primarily from the Lagrangian.

Essentially, we see each cell in the grid as a newly-moved particle in
a Lagrangian system. To update a given cell, $\vec{x}_G$, we first
look up the velocity at the grid center. Tracing this velocity
backwards, we find where the "particle" originated from and take the
value from that location to write to the current cell. The method of
backtracing can be as simple as a single step of forward Euler, which
has been discussed previously. For increased stability we use a
second-order Runge-Kutta method, as recommended in section 3.1 of
\cit{bridson2008fluid}. With Runge-Kutta we first take a half-step
against the direction indicated by the current cell to get an
intermediate position: $\vec{x}_{mid} = \vec{x}_G - \frac{1}{2}\Delta
t \vec{u}\left(\vec{x}_G\right)$. We then use the velocity there to
arrive at the final value.

After advecting, adding a force such as gravity is as simple as
modifying both components of the MAC grid in the desired
direction. After this is done, we are ready for the pressure
calculations.

\section{Incompressibility}
We arrive now at the part of the simulator that most gives rise to
fluid-like behavior of the interface.

Recall that we want to solve equation \vref{eq:pressure} while
satisfying equation \vref{eq:divergence}. We use u and v as the
horizontal and vertical components of velocity in the mac grid, using
half-indices to indicate whether we are sampling at the top, bottom,
left or right edge of the cell in question. Thus, the horizontal
component at cell $(i,j)$ is designated $u_{i-\frac{1}{2},j}$. The
formulas for the pressure update are as follows:
\begin{align}
u_{i+\frac{1}{2},j}^{n+1} &= u_{i+\frac{1}{2},j} - \Delta t\frac{1}{\rho}\frac{p_{i+1,j}-p_{i,j}}{\Delta x},\\
v_{i,j+\frac{1}{2}}^{n+1} &= v_{i,j+\frac{1}{2}} - \Delta t\frac{1}{\rho}\frac{p_{i,j+1}-p_{i,j}}{\Delta x}
\end{align}
Notice in particular how the pressure is calculated at cell centers by
using the velocity at cell-edges for the central difference. This is
the central purpose of the MAC grid.

How, then, do we calculate the pressure? This portion of the
simulation is mathematically simple, but quite complex in
implementation, since we are interested in solving a large linear
system of equations.  Fortunately, a lot of the groundwork has been
done for us already. We leave the actual solving of the system to an
imported mathematical library and focus simply on properly formulating
the problem.

We may write the problem of finding the pressures in the grid in the
following simple matrix-vector form:
\begin{equation}
Ap = b
\end{equation}
$p$ is what we are interested in finding, the domain-wide grid of
pressures. $A$ is a matrix of coefficients to these pressures: A has a
row for each fluid-filled cell, with each of these rows containing
non-zero values only in the cells corresponding to the cell in
question and its six neighbours. In this way, $Ap$ represents the
unknown pressure for each cell in the grid. The right hand side of the
equation, $b$, is another domain-sized matrix containing the negative
divergences of each cell. Recall that we are trying to minimize
divergence across the fluid. This is in fact the entire point of the
pressure update. The pressures we are trying to find are exactly the
pressures that, when added to the velocity field, will keep the field
divergence free.  We won't go into the details of the construction of
the $A$-matrix. It is technically straightforward since we know
exactly where our fluid currently is.

The $b$ matrix on the right hand side of the equation is constructed
from the component velocities as follows:
\begin{lstlisting}
scale = 1 / dx
loop over i,j where phi(i,j) < 0
   rhs(i,j) = -scale * (u(i+1,j)-u(i,j)
                       +v(i,j+1)-v(i,j));
\end{lstlisting}
Of course, we are only interested in the cells which contain fluid,
which in our representation is everything on the negative side of the
zero-level isosurface.

Finally, we relegate the solving of this system to an implementation
of the Preconditional Conjugate Gradient method. This is an iterative
method, the precise implementation of which lies somewhat outside the
scope of this paper. In short, the result of applying this method to
our constructed linear system yields the grid of pressures. Applying
this to the velocity field should purge it of divergence, thereby
completing this step of the algorithm.

Finally, using this divergence free velocity field, we advect the
values of the signed distance field. To this end, we use the same
Runge-Kutta method as described in the advection of the velocity
field. Of course, having done this we need to reinitialize the signed
distance field, as it likely no longer lives up the the requirements
described in chapter \vref{chap:sdf}.

Having done this, we are ready to start a new iteration afresh, and
thus we may iteratively evolve our interface in full accordance with
the Navier-Stokes equation.

\section{Conclusion}
While there are other methods for the simulation of fluids, the
mathematical simplicity of the equations used here should show that
the level set method lends itself particularly well to this sort of
physical simulation. Other similar methods use particles to track the
surface of the fluids, but these can prove expensive and relatively
inaccurate.

Care needs to be taken, however, when using level sets. If features
arise in the fluid which are smaller than the size of a grid-cell,
they may disappear complettely as the grid is unable to represent
them. Therefore, to preserve a natural-seeming motion one should make
sure that no accuracy smaller than the cell size is required.

An implementation of the level set based fluid simulator has been
incorporated into the joint level set project. At the time of writing,
however, the implementation is quite unstable. Currently, little
regard is being given to the size of the time-step for each
iteration. This causes the simulation to "blow up" before any useful
simulation has been done! Fixing the time-step to something sensible
is the obvious next step for the algorithm. Section 3.3 of
\cit{bridson2008fluid} describes a method of limiting it to maximize
stability.

%%% Local Variables: 
%%% mode: latex
%%% mode: auto-fill
%%% TeX-PDF-mode: t
%%% TeX-master: "../master.tex"
%%% End: 
