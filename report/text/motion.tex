

In contract to section \vref{sec:extVel}, this section describes
motion using a internally generated velocity field.

\newpage

\subsection{Grow/Shrink}

One of the most simple things we can do, is to move the interface in
the normal direction $\vec{N}$ using a constant $a$, efficiently
scaling the iso-surface.

From equation \vref{velocity:levelseteq} we have:
\begin{equation}
  \phi_t + \vec{V}\cdot \nabla \phi = 0
\end{equation}
Since $a\vec{N}$ have the same direction as $\nabla{\phi}$, we have the following:
\begin{align*}
  a\vec{N}\cdot\nabla\phi &=
  a\frac{\nabla\phi}{|\nabla\phi|}\cdot\nabla\phi \\
  &= a\frac{|\nabla\phi|^2}{|\nabla\phi|} 
  = a|\nabla\phi|
\end{align*}

With corresponds to this level set equation:

\begin{equation}
  \phi_t + a |\nabla \phi| = 0
\end{equation}\label{eq:lsgrow}

At this point, the point of the signed distances field is clear, if
$\phi$ is a SDF, then $|\nabla\phi| = 1$. This means that we can solve
(\ref{eq:lsgrow}) very simply using the forward euler method (see
equation \ref{eq:fweuler}).

This corresponds to the following code:

\begin{lstlisting}
for(unsigned int x=0; x<width; x++) {
    for(unsigned int y=0; y<height; y++) {
        phi(x,y) += -a;
    }
}
\end{lstlisting}

\todoPtx{Billede?}

\subsection{Mean-Curvature}

\subsection{Morph}

\subsection{CFG condition (Stability)}



%%% Local Variables: 
%%% mode: latex
%%% mode: auto-fill
%%% TeX-PDF-mode: t
%%% TeX-master: "../master.tex"
%%% End: 
