

In contract to section \vref{sec:extVel}, this section describes
motion using a internally generated velocity field.

\newpage

\subsection{Grow/Shrink}

One of the most simple things we can do, is to move the interface in
the normal direction $\vec{N}$ using a constant $a$, efficiently
scaling the iso-surface.

From equation \vref{velocity:levelseteq} we have:
\begin{equation}
  \phi_t + \vec{V}\cdot \nabla \phi = 0
\end{equation}
Since $a\vec{N}$ have the same direction as $\nabla{\phi}$, we have the following:
\begin{align*}
  a\vec{N}\cdot\nabla\phi &=
  a\frac{\nabla\phi}{|\nabla\phi|}\cdot\nabla\phi \\
  &= a\frac{|\nabla\phi|^2}{|\nabla\phi|} 
  = a|\nabla\phi|
\end{align*}

With corresponds to this level set equation:

\begin{equation}
  \phi_t + a |\nabla \phi| = 0
\end{equation}\label{eq:lsgrow}

At this point, the point of the signed distances field is clear, if
$\phi$ is a SDF, then $|\nabla\phi| = 1$. This means that we can solve
(\ref{eq:lsgrow}) very simply using the forward euler method (see
equation \ref{eq:fweuler}).

This corresponds to the following code:

\begin{lstlisting}
for(unsigned int x=0; x<width; x++) {
    for(unsigned int y=0; y<height; y++) {
        phi(x,y) += -a;
    }
}
\end{lstlisting}

\todoPtx{Billede?}

\subsection{Mean-Curvature}

Another interesting thing we can do with a level set, is to move the
interface in the normal direction with a velocity proportional to its
curvature (this is called mean-curvature). The effect of this is to
soften or sharpen the interface.

To do this we need the following velocity field:
\begin{equation}
  \vec{V} = -b\kappa\vec{N}
\end{equation}

Where $\kappa$ is the curvature, and $b>0$ is a constant describing the
speed of the motion. \todoPtx{Explain why $b>0$?}

This corresponds to the level set equation: 
\begin{equation}
  \phi_t -b\kappa|\nabla\phi| = 0
\end{equation}

This is a parabolic equation, so to discretize it we need to use a new
approach. 

We start by exploring the curvature $\kappa$. It is defined as:
\begin{equation}
  \kappa = \nabla\cdot\left(\frac{\nabla\phi}{|\nabla\phi|}\right)
\end{equation}

When $\phi$ is a SDF, then $|\phi|=1$, which simplifies our equation
to: $\kappa = \nabla^2\phi$. Which is the laplacian operator: $\kappa =
\Delta\phi$. In this context, $\Delta\phi$ is defined as:
\begin{equation}
  \Delta\phi = \phi_{xx} + \phi_{yy}
\end{equation}

$\phi_{xx}$ and $\phi_{yy}$ can be solved using central difference:
\begin{equation}
  D^+_xD^-_x\phi 
  % \frac{\partial^2\phi}{\partial x^2} % Should we use \partial or D?
  \approx
  \frac{\phi_{i+1}-2\phi_i+\phi_{i-1}}{\Delta x^2}
\end{equation} % eq 1.9 s. 12

Again the forward Euler method can be used, but we have to use a small
time step according to \citbook{osher2002level}{page~44}:
\begin{equation}
   \Delta t \left(
     \frac{2b}{(\Delta x)^2} +
     \frac{2b}{(\Delta y)^2}
   \right) < 1
\end{equation}

If we respect these, then the  implementation is straight forward:

% using formula 1.9 on page 12 and
% using formula 2.7 on page 21.
% using formula 4.11 on page 45

\begin{lstlisting}
for(unsigned int x=1; x<width-1; x++) {
    for(unsigned int y=1; y<height-1; y++) {
        Vector<2,float> g = sdf->Gradient(x,y);

        const float dx = 1.0;
        float phi_xx = (phi(x+1,y) - 2*phi(x,y) + phi(x-1,y))/(dx*dx);
        float phi_yy = (phi(x,y+1) - 2*phi(x,y) + phi(x,y-1))/(dx*dx);

        float kappa = phi_xx + phi_yy;
 
        phi(x,y) += kappa * a; //mean curvature
    }
}
\end{lstlisting}

\todoPtx{pics!}

\subsection{Morph}

$\phi_t = \phi + (\phi - \phi_b \cdot -a)$

\begin{lstlisting}
for(unsigned int x=0; x<width; x++) {
    for(unsigned int y=0; y<height; y++) {
        phi(x,y) += (phi(x,y) - phi2(x,y)) * -a; //morph
    }
}
\end{lstlisting}

\todoPtx{picture here!}

% \subsection{CFG condition (Stability)}
% WTF is this?

%%% Local Variables: 
%%% mode: latex
%%% mode: auto-fill
%%% TeX-PDF-mode: t
%%% TeX-master: "../master.tex"
%%% End: 
