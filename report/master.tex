\documentclass[a4paper, oneside]{memoir}
\usepackage[utf8]{inputenc}
%\usepackage[T1]{fontenc}
\usepackage{verbatim}
\usepackage{graphicx}
\usepackage{listings}
\usepackage{listing}
\usepackage{amsthm}
\usepackage{amsfonts}
\usepackage{amsmath}
\usepackage{subfig}
\usepackage{varioref}
\usepackage{color}
\usepackage{pgf}
\usepackage[colorinlistoftodos]{todonotes}
%\usepackage[colorlinks, linkcolor=blue, citecolor=red, urlcolor=brown]{hyperref}
\usepackage{hyperref}
\hypersetup{pdfborder=0 0 0} % Go away you ugly boxes!
% ^- farver gør det meget nederen at udskrive en s/h kopi, da alle "se
% fig 1.1 på side xx" bliver grå! (ptx)

\usepackage{natbib}

%Individual todos
\newcommand{\todoPtx}[2][]{\todo[color=red,    #1]{Ptx: #2}}
\newcommand{\todoCpvc}[2][]{\todo[color=magenta, #1]{Cpvc: #2}}
\newcommand{\todoSean}[2][]{\todo[color=yellow,   #1]{Sean: #2}}
\newcommand{\todoHave}[2][]{\todo[color=green,  #1]{Have: #2}}
\newcommand{\todoVester}[2][]{\todo[color=lime,  #1]{Vester: #2}}

\renewcommand{\ttdefault}{pcr}


\lstset{language=C}
%\lstset{backgroundcolor=listinggray}
%\lstset{backgroundcolor=\color{listinggray}}
%\lstset{linewidth=90mm}
%\lstset{frameround=tttt}
%\lstset{frameround=trbl}
%\lstset{labelstep=1}
%\lstset{keywordstyle=\color{blue}\textbf}
\lstset{keywordstyle=\textbf}
%\lstset{moredelim=[is][\ttfamily]{|}{|}}
\lstset{basicstyle=\ttfamily \footnotesize}
%\lstset{basicstyle=\ttfamily \small}
%\lstset{commentstyle=\ttfamily}
\lstset{commentstyle=\normalfont \textit}
\lstset{stringstyle=\bfseries}
\lstset{showstringspaces=false}
%\lstset{numbers=left,numberstyle=\ttfamily \small}
%\lstset{breaklines=true}


% \lstset{
%   basicstyle=\ttfamily \footnotesize,
%   keywordstyle=\color{red}
% }


% Dokumantation:
% http://www.tex.ac.uk/tex-archive/macros/latex/contrib/todonotes/todonotes.pdf
% Vigtigste kommandoer:
% \todoVester[inline]{text}
% \missingfigure{A illustration of how peers are placed on the ring}
% \todo{Introduction to this section}
% \todo[inline]{Someone write this}

% replaces cite. Use \cit instead!
\newcommand{\cit}[1] {\cite{#1}}

% cite with page or section ref
\newcommand{\citbook}[2] {\citep[#2]{#1}}

% when defining a new term
\newcommand{\dit}[1] {\textit{#1}}

% \image{image, scale, caption, label}
\newcommand{\image}[4]{
  \begin{figure*}[!htb]
    \centering
    \includegraphics[scale=#2]{imgs/#1}
    \caption{#3}
    \label{#4}
  \end{figure*}}
% ^- Det her går aldrig godt, man har aldrig brug for at ændre flere
% ting end der er muligt med denne slags makroer. (ptx)

\title{Level Sets!!}
\author{
 \\
  Martin Have (20051456)\\
  \url{have@cs.au.dk}\\
  \\
  Peter Kristensen (20051866)\\
  \url{ptx@cs.au.dk}\\
  \\
  Mikkel Vester (20053229)\\
  \url{vester@cs.au.dk}\\
  \\
  Christian P. V. Christoffersen (20050879)\\
  \url{cpvc@cs.au.dk}\\
  \\
  Sean Geggie (20052203)\\
  \url{geggie@cs.au.dk}\\
  \\
  University of Aarhus
}

\begin{document}

\maketitle{}


\missingfigure{Her er et billede af et morph ting}

% Hvem laver hvad.
\begin{comment} % orgtbl-mode er win
|----------------------------+----------+--------|
| Afsnit                     | Indhold  | Hvem   |
|----------------------------+----------+--------|
| Intro                      | ch. 1,2  | have   |
| - noget om lys/raycasting  |          | ptx    |
| Build SDF / Discretization | artikel? | vester |
| Reinitialize               | ch. 7    | cpvc   |
| Motion                     | ch. 4,6  | ptx    |
| Externally gen. vf.        | ch. 3    | geggie |
| Godunov                    | ch. 5    | cpvc   |
|----------------------------+----------+--------|
\end{comment}

\newpage

\tableofcontents{}
\listoftodos
\chapter{Introduction}
\label{chap:introduction}
  When simulating physical phenomena one must choose a model which fits
the phenomena being modeled. The level set method (LSM), is a method
for modelling phenomena that can be described in terms of moving level
sets.
%
A level set in itself is only a mathematical function that groups
variables which have the same function value. It describes a
parametric function in one dimension higher than its original domain,
gaining the ability to describe more than one not colocated level set
in the same structure.
%
A two dimensional level set is known as a level curve or isocontour,
which uses two dimensions to describe curves. We encounter isocontours
in everyday life as pressure information in the weather forecast and
terrain elevation on maps.
%
A three dimensional level set is
known as a level surface or isosurface, and is used, as suggested be
it's name, to describe surfaces.  A general a level set in n
dimensions is used to describe n-1 dimensional objects. Mathematically
a level set is define as follows:

\begin{equation}
\{ (x_1,...,x_n) \} | f(x_1,...,x_n) = c
\end{equation}

In words this means that all values of $v_i$ leading to a function
value of $c$ defines the level set $c$.
%
The level set in its own right does not make a simulation. Unless it
is evolved over times it stays the same. To evolve the level set over
time, partial differential equations (PDEs) describing the physical
phenomena needs to be solved to move the level sets.
%
Typical phenomena modeled by the LSM includes: computational fluid
dynamics (CFG), and fire and explosion simulation. 

In this report, we will explain what the Level Set method is and what
its applications are. Furthermore, we will provide code examples of
how to implement the mathematical formulas.

In figure \vref{fig:heightmap} we see two figures describing a two dimensional and
three dimensional view of an island. Figure \vref{fig:isocontour-2d} describes an
isocontour map of heights where the color indicates the height of each
point. The brighter the color the higher the point. Figure
\vref{fig:isocontour-3d} shows the corresponding 3d view.

\begin{figure}[h]
\begin{center}
  \subfloat[Two dimensional view of an isocontour map]{
    \includegraphics[width=0.5\textwidth]{imgs/226171_226171.png}
    \label{fig:isocontour-2d}
  }
  \subfloat[Three dimensional view of figure \ref{fig:isocontour-2d}]{
    \includegraphics[width=0.5\textwidth]{imgs/226170_226170.png}
    \label{fig:isocontour-3d}
  }
\end{center}
\caption{Illustrations of heightmap, from: Intel Array Visualizer
  Gallery, \cit{intel}}
\label{fig:heightmap}
\end{figure}


For a level set representation there is an underlying function
describing the topology. Each contour corresponds to a set of points
in this function with the same value, thus we need a function which
produces a field of values. In heightmaps, it would be a function from
$(x,y)$ coordinates to a height. In general, any function producing a
field of values is sufficient.  

One function that has the desired characteristics is a function known
as the signed distance function $\phi$.


\section*{Outline of the report}

In chapter \vref{chap:sdf}, we give an in depth look on the signed
distance function, describe what kinds of mathematical operations we
have in our toolbox and describe the important reinitialize function,
section \vref{sec:reinitialize}.

We have individually made a number of extensions to the basic level
set implementation. These extensions should show the versatility of
the method and show some practical applications.

In chapter \vref{chap:extensions}, we look at these extensions. Each
section has been written by the person responsible for it.
In section \vref{sec:narrowband}, for efficiency, we improve the
signed distance implementation to use the narrow band technique. 
In section \vref{sec:cuda} we improve the reinitialization step by
implementing it on the GPU.
In section \vref{sec:segmentation}, we look at how to implement
segmentation algorithms which can be used in mediacal imaging. 
In section \vref{sec:fluid}, we implement a fluid solver for computer
graphics using the level set method. 



%%% Local Variables: 
%%% mode: latex 
%%% mode: auto-fill 
%%% TeX-PDF-mode: t 
%%% TeX-master: "../master.tex" 
%%% End:



\chapter{SDF}
\label{chap:sdf}
I this chapter we will explain some of the more detailed aspects of
the signed distance field we use to represent $\phi$, the signed
distance function. It is in this field we keep, manipulate and work on
our zero-level set. 

We will explain how it is created and maintained, and some of the
different functions we apply to the field to make it morph, grow,
shrink or move.



\section{Building the SDF}
\label{sec:initialization}
In our solution we import the initial contours by loading a black and
white image and analyse it to create the SDF matching it. 

To generate the SDF we start by constructing two Cartesian grids of
the same sizes as the image. One to hold the distances to the contour
on the outside of the object, and one for the inside. The ``outer
grid'' is populated with (0,0) if the pixel is dark and
(``infinite'',``infinite'') if it's white, and opposite for the inner
grid. This way each entry in the grid has a vector.

The next step is to calculate the actual distances to the iso-contour
for each pixel. We first traverse the image pixel by pixel from top
left corner to the lower right. For each row in the grid we go from
left to right and for each pixel calculate the length of our vector in
this entry against all the neighboring pixels vectors. The distance is
the vector length of the vector created with the x,y values in the
grid. If the distance in a neighbor, plus the offset to the current
pixel, is smaller than the one already in the given pixel, it is
substituted with the new and smaller one. This way we keep the 0
values inside the object, and use this value to propagate out to the
pixels we visit afterwards.

\begin{figure}[htb]
  \centering
  \includegraphics[scale=0.8]{imgs/sean-mikkel.png}
  \caption{Building the Signed Distance Function table, pixel by pixel from an image.}
  \label{build_sdf:fig:sean-mikkel}
\end{figure}

When we reach the end of a row we go back again in the same row, and
now check the same pixels, this time checking the pixel to the right
of it, also filling out distances to the left of objects. The
navigation in the grid can be seen in
figure \vref{build_sdf:fig:sean-mikkel}.

We now have distances below all objects in the image. If we now run
the same pass again, just starting from the bottom and moving up, we
have a vector at each pixel with the offset to the nearest contour.

Doing the same with the opposite starting values will create a
distance grid with the values from inside objects in the
image. Subtracting the distance of the vectors in the second grid
from the first will give us an $\phi$-value for each pixel, and we can
now save these as our $\phi$.

This SDF will have to be reinitialized numerous times afterwards to
complete it, as the contour should be a smooth line instead of a 1-pixel
wide line following pixels completely. This will be covered in section \vref{sec:reinitialize}. When the SDF has been reinitialized it will become as smooth as the one depicted in figure \vref{introduction:fig:cartesiangrid}.





\pagebreak
\section{Reinitialization}
\label{sec:reinitialize}
% -*- mode: latex; mode: auto-fill; coding: utf-8; -*-

The main advantage of representing an implicit counture or surface as
a signed distance field, is that the length of the gradient is
one. This is also exactly what defines a signed distance field.

When constructing or manipulating a signed distance field defined on a
Cartesean grid, the result is not always signed distance field. So to
enable further calculations or iterations of an algorithm the result
must be turned into a new signed distance field that reflexts the
changes done by the calculations. This process is called
\dit{reinitialization} of the level set, and can be done several
different ways.

Algorithms to reinitializing signed distance fields focuses on
reinitializing the whole domain, and at the same time keeping the zero
level set as fixed as possible. This means that the process disrupts
data outside the zero level set, which depending on the type of
phemomena we are modelling can be problematic. For the problems we are
modelling this is not an issue, and is therefore not of relevance here.

Before diving into the algoritms a good question is, how often must the
level set be reinitialized? There is no good answer to this question,
because it depends on how rapidly the contour is changing. In our work
we have been reinitializing after ever change, which ensures that this
is not a source of error.

The two most used algoritmic approched to reinitialization are:
geometricly methods that calculations distances to the contour or surface,
and PDE based methods that numerical approximates solutions to the Eikonal
equation: $||\nabla \phi|| = 1$ \citbook{bridson2008fluid}{page~89}.

When initializing the implicit contour in section \vref{sec:initialization}
we used an algorithm that geometricly calculated the distance to the
contour. So the natural choise should be also to use this algorithm for
reinitializing, but because the algorithm does not calculate the
signed distance function precise enough, we cannot use this algorithm
when reinitializing. Furthermore if we had used this type of
algorithm, the we would have had to construct the contour before
invoking the algorithm. Constructing the contour is very expensive,
and is something that should be avoided at almost all cost when using
the level set method.

Instead we have chosen to use PDEs to solve the Eikonal
equation. Solving the Eikonal equation can also be done in different
ways, which we will describe in the next subsections.

\subsection{The PDE way of reinintalizing}
When using the PDE way of reinitialing the level set, we solve the
following PDE \citbook{osher2002level}{page~65-66}:

\begin{equation}
\label{eq:reinit}
\phi_t + S(\phi_0)(|\nabla \phi| - 1) = 0
\end{equation}

Where $\phi$ is the SDF being evolved as a PDE, and $\phi_0$, is the
initial SDF giving as input to the reinitialization algorithm, which
has not been altered by the process of solving the PDE. The function
$S(\phi_0)$ gives the sign of the SDF, like the following function,
described in \citbook{osher2002level}{page~66}:
%\citbook{article:FLLSM}{page~419}:

\begin{equation}
\label{eq:s1}
S(\phi) =
\begin{cases}
-1 &\mbox{ if } \phi < 0, \\
 0 &\mbox{ if } \phi = 0, \\
 1 &\mbox{ if } \phi > 0.
\end{cases}
\end{equation}

When using this approach of reinitializing the SDF, then the grid
points in $\phi$ that are nearest to the contour is reinitialized
first, and then propergated in the normal direction from the zero
level set, hereby reinitializing the grids points in layers each
iteration.

This algorithm is relatively slow if all grid points needs to be
reinitialized, because it only reinitializes one layer in each
iteration when solving the PDE. This means that the PDE, on a
two-dimensional domain, must be iterated
$\sqrt{width^2 \times height^2}$ times to make sure that the algoritm
has reinitialized the whole domain.

But because the algorithm has the property of reinitializing the SDF
in layers it is of special interest in reguards to performance when
using a narrow band level set as described in section
\vref{sec:narrowband}. Here only three or four layers around the SDF needs
to be reinitialized, making the algorithm ideal for the narrow band
approach.

\subsection{The sign function S}
Because equation \ref{eq:reinit} is a hyperbolic PDE, we need to use a
smeared out version of equation \eqref{eq:s1}. One way of smearing the
function is to using equation \eqref{eq:Sphi0}.

\begin{equation}
\label{eq:Sphi0}
S(\phi_0) = \frac{\phi_0}{\sqrt{\phi_0^2 + (\Delta x)^2}}
\end{equation}

The difference between equation \eqref{eq:s1} and equation
\eqref{eq:Sphi0}, can more easily be seen be looking at a plot of the
two functions, as depicted in figure \ref{fig:s-graph}.

\begin{figure}[h]
\begin{center}
  \subfloat[Plot of equation \eqref{eq:s1}]{
    \includegraphics[width=0.5\textwidth]{imgs/S0.pdf}
    \label{fig:fake1}
  }
  \subfloat[Plot of equation \eqref{eq:Sphi0}, $\Delta x=1$]{
    \includegraphics[width=0.5\textwidth]{imgs/S1.pdf}
    \label{fig:fake2}
  }
\end{center}
\caption{Illustrations of S}
\label{fig:s-graph}
\end{figure}

%\subsection{CFL-condition}


\pagebreak
\subsection{Implementation}
The implementation of how to solve the PDE from equation \eqref{eq:reinit},
uses the describtion Godunov's scheme from \citbook{osher2002level}{page~58}
in the spatial dimensions, and a forward Euler in time, described in
formular (1.3) in \citbook{osher2002level}{page~10}.

\begin{lstlisting}[language=c++]
    Tex<float> phi0 = GetPhi(); Tex<float> phin = GetPhi();
    for (unsigned int i = 0; i<iterations; i++) {
        for (unsigned int x = 0; x < width; x++) {
            for (unsigned int y = 0; y < height; y++) {
                float xy = phi(x, y);                
                float phiXPlus = 0.0f;
                float phiXMinus = 0.0f;
                float phiYPlus = 0.0f;
                float phiYMinus = 0.0f;        	
                if (x != width-1) phiXPlus  = (phi(x+1, y) - xy);
                if (x != 0)       phiXMinus = (xy - phi(x-1, y));
                if (y !=height-1) phiYPlus  = (phi(x, y+1) - xy);
                if (y != 0)       phiYMinus = (xy - phi(x, y-1));
        	
                float dXSquared = 0;
                float dYSquared = 0;
                float a = phi0(x,y);
                if (a > 0) {
                    // formula 6.3 page 58
                    float max = std::max(phiXMinus, 0.0f);
                    float min = std::min(phiXPlus, 0.0f);
                    dXSquared = std::max(max*max, min*min);
                    max = std::max(phiYMinus, 0.0f);
                    min = std::min(phiYPlus, 0.0f);
                    dYSquared = std::max(max*max, min*min);
                } else {
                    // formula 6.4 page 58
                    float max = std::max(phiXPlus, 0.0f);
                    float min = std::min(phiXMinus, 0.0f);
                    dXSquared = std::max(max*max, min*min);
                    max = std::max(phiYPlus, 0.0f);
                    min = std::min(phiYMinus, 0.0f);
                    dYSquared = std::max(max*max, min*min);        				
                }
                float normSquared = dXSquared + dYSquared;           
                float norm = sqrt(normSquared);

                // Using the S(phi) sign formula 7.6 on page 67
                float sign = phi0(x,y) / sqrt(phi0(x,y)*phi0(x,y) + 1);
                float dt = 0.3; // A stabil CFL condition
                phin(x,y) = phi(x,y) - sign*(norm - 1)*dt;
            }
        }
        for (unsigned int y=0; y<height ; y++)
            for (unsigned int x=0; x<width; x++)
                phi(x,y) = phin(x,y);

    }
\end{lstlisting}


\section{CSG (Union, Intersection, Minus)}
\subsection{Union}
\subsection{Intersection}
\subsection{Minus}

\section{Externally generated velocity field}
%% -*- mode: latex; mode: auto-fill; -*-
% -*- set-buffer-file-coding-system: utf-8; -*-

\subsection{The Eulerian and Lagrangian point of view}
% Through the report, the examples and theory will be presented in the
% 2D .. as this is easy to visualise on paper. After each of the
% describtions this will be extended to 3D, and all formules for the 3D
% calculations will be explicit written so the reader knows all the
% details which is later use to implement the simulation.

When descriping motion of a substance, no mather if it is on fluid
(gass or liquid) or solid state, we have two well known models which
captures the problem: the Eulerian and the Lagrangian viewpoints. To
understand the difference between to the viewpoints we imagine how we
could measure the movement of e.g. a fluid. In the Eulerian viewpoint
we would place measuring devices at fixed points in the fluid, and
continues sample the velocity of the fluid, the measured value is
taken as a avarage for an area. For conviniences the measuring devices
are almost always placed in a uniform grid, with square areas as
illustrated in figure \ref{fig:eulerian}.

\begin{figure}[h]
  \centering
  \subfloat[Eulerian]{
    \includegraphics[width=0.5\textwidth]{imgs/eulerian.png}
    \label{fig:eulerian}
  }
  \subfloat[Lagrangian]{
    \includegraphics[width=0.5\textwidth]{imgs/lagrangian.png}
    \label{fig:lagrangian}
  }
  \caption{The eulerian and the lagrangian viewpoints.}
  \label{fig:eulerian-lagrangian}
\end{figure}

In the lagrangian viewpoint we let the measuring devices move along
the current in the fluid and take measurements at different locations
in the fluid. Here we imagen that the measuring device is part of the
fuild or a particle that the fluid moves around. Figure
\ref{fig:lagrangian} shows how the current has moved the devices as
time has progressed.

These two viewpoints are tightly coupled to the way the simulation data
is represented and visuliazed. Lets say that we are interested in
visualizing the surface of the substance.
%
The Lagrangian data is keept as points in space and moved around by
updating the position of the points. This can be visualizes by
imposing that the points define a surface and rendered as geometry
e.g. triangles between the points.
%
The eulerian viewpoint is seen as a 3D grid. Each cell in the grid has
a density that describes how much substance the cell contains (0-100\%).

Whiled this example focuses on measuring movement, the two different
viewpoints can be imposed on all kinds measurements.
%
When modelling a fluid, the Eulerian viewpoint is often
enough. But for more advanced simulations of fluid or solid,
Lagrangian is often used so the underlaying math equations are simpler
to describe.

%And example of this is: When modelling ...

%The are algoritmens that converts between the two viewpoints: .. . And
%as such nothing prevents us from using both viewpoint when defining
%the mathematically models that drives the simulation. As can be seen
%from the conversion algorithms, this envolves a fair amount of
%math. So to make the models more readable, it is advised to model
%in one of the domains, and only when optimizing the formulaes for
%performance, introduce conversions.

%In all the mathematically models that are developed in this repor the
%Lagrangian viewpoint is will be used.

%ref: fluids book, michael bang figures,

% \subsection{Data models}
% In the different parts of the simulator we use two different data
% model which are coupled tightly the the two points of view just
% reviewed. The two data models are refered to as the vertex/primitive
% model and the volumetric model. Each of which will now be elaborated.

% \subsubsection{The Vertex/Primitive model}
% When using the Lagragian viewpoint, we call the data model for a
% vertex/primitive model. In this model we have a vertex pool
% constituting which points are available, and some primitives, which
% referes to the vertices and binds these togther.

% \textbf{Primitives}
%  Fx as triangle, quads or tetrahedras.

% \subsubsection{The volumetric model}
% When using the Euler viewpoint in 2D, vi call the grid a texture and
% the cell for pixels. As we moves to 3D pixels are called voxels as
% they cover a volume.

% \textbf{density field}
% The voxels has value from 0 to 1. Which is to be interpreted as the
% procentage of solid the voxel contains.
% \textbf{Vector fields} \url{http://en.wikipedia.org/wiki/Vector_field}
% \textbf{Tensor fields} \url{http://en.wikipedia.org/wiki/Tensor_field}
% \textbf{energi density field}
% \url{http://users.powernet.co.uk/bearsoft/Field.html}

% \textbf{ISO Surface (level sets)}
% Each voxel is set to zero when it contains the surface of the solid
% and 1 all other places. \url{http://en.wikipedia.org/wiki/Isosurface}

% \textbf{distance field}
% Nearly the same as an iso surface. That is it zero on the surface, but
% instead of being 1 elseware, the voxels contains a value corresponding
% to the shortests distance to the surface.

% \textbf{signed distance field (SDF)}
% A distance field, but with a sign determining if the voxel is inside or
% outside the solid. So fx the distances inside the solid could be
% negative, and the voxels outside could be positive.

% The density field can be converted into a signed distance field or iso
% surface and rendered using standard volume rendering techniques.

% \subsubsection{Vertex/Primitive contra Volumetric}
% For a 3D graphics framework like OpenGL or DirectX, the
% vertex/primitive data model is the most convinient as data can be
% mapped directly to visual output.
% %
% But in some situations the vertex/primitive model is not the first
% choise. When representing the data as volumetric data the data model
% imposes a grid structure on the data, making some thing easier to
% do. Fx when doing collision detection the volumetric model has
% advantages.

% \subsection{Getting geometry input data}

% \subsubsection{X-ray images}
% The main source of surface and body (geometry) data comes from 3D X-ray
% image scans. X-ray images are volumetric data in black and white where
% each voxel has a value which describes the density in that voxel.
% When taking an X-ray image of bones or teeth, X-ray pulses are
% shot through the body with radiographic film behind. The bones or
% teeth absorb the most photons by the photoelectric process, because
% they are more electron-dense. The X-rays that do not get absorbed turn
% the photographic film from white to black, leaving a white shadow of
% bones and teeth on the film.

% \subsubsection{Segmentation}
% Patients are scanned and the dentist or other personal prepare
% the X-ray images for simulation by segmenting out the tooth that is to
% be operated upon in the simulator. As an example the segmentation can
% be done in the software ITK-Snap as showen in figure
% \ref{fig:itk-snap}, in the figure the red tooth has been segmented by
% first applying an automatic segmentation algorithm and afterwards rofly
% adapted by hand. The tooth segmented with green, has only been
% automaticly segmented, which also shows as the segmentation flows into
% the tooth next to it.

% \begin{figure}[H]
%   \centering
%   \includegraphics[width=14cm]{./images/itk-snap.png}
%   \caption{Segmentation in ITK-Snap.}
%   \label{fig:itk-snap}
% \end{figure}

% After the segmentation has taken place, the 3D image is exported, into
% a unsigned byte density field where voxels in the model (maked as red
% in the figure) get value 255, and voxels outside the model are set to
% 0. A guide of how to use ITK-Snap is provide in appendix \ref{guide:itk-snap}.

% \section{The simulator parts}
% The full simulator is made up of several different part or sub
% simulators. Each of the sub simulators does different things, and must
% therefore be designed specifictly for the job they need to handle. As
% mentioned in the introduction, we have identified the following
% different sub simulators. Now is that time to specify what they do and
% how we think it can be done.

% \subsection{Cutting tissue}
% As in: Virtual Reality
% Heart\footnote{\url{http://www.systematic.dk/om+os/innovation}}. I
% vertex based model and tools like knife and tweezers.

% \begin{figure}[H]
%   \centering
%   \includegraphics[width=14cm]{./images/heart-sim.png}
%   \caption{Screenshot from: The Heart Simulator.}
%   \label{fig:heart-sim}
% \end{figure}

% Picture from: \url{http://www.jespermosegaard.dk/Projects}

% \url{http://www.alexandra.dk/nyhedsbrev/to/05/november05/status_cavi.htm}
% \url{http://www.cavi.dk/projects/surgical_simulation.php}
% \url{http://www.daimi.au.dk/~mosegard/medVisSim/}
% \url{http://cg.alexandra.dk/2009/04/30/simulation-of-congenital-heart-surgery/}
% \url{http://www.daimi.au.dk/~cfpc/projects/cadiovas/cadiovas_summary.htm}

% \subsection{Dividing into smaller pieces}
% This job, has two distinct sub simulators, one for drilling and on for
% breaking. It has been divide this way, as the drilling is easiest done
% on a volumetric data model while the breaking is done using linear
% elastic model which are based on deforming the body making a vertex
% model more preferable.

% \subsubsection{Drilling}
% As the tooth have been segmented and exported from itk-snap, this data
% can be used directly in this sub simultor.
% %
% Drillng as done in the The Visible Ear
% Simulator\footnote{\url{http://www.alexandra.dk/ves/}}, are done
% directly on volumetric data. It works by simulating the drill head as
% a sphere, and when rotation the simulator modifies the voxels by
% removing material where the sphere are located.

% A possibility is to base the hardness of the material on the intensities
% of the material from the X-ray images.

% \begin{figure}[H]
%   \centering
%   \includegraphics[width=14cm]{./images/ear-sim.png}
%   \caption{Screenshot from: The Visible Ear Simulator.}
%   \label{fig:ear-sim}
% \end{figure}

% Picture taken from:
% \url{http://cg.alexandra.dk/category/visible-ear-simulator/}.

% To visualize the data, standard volumetric rendering techniques like
% ray tracing \ref{ray_tracing} or photon mapping \ref{photon_mapping}
% can be imployed. This is also what is done in The Visivle Ear
% Simulator.

% As all these techniques are implemented and used in The Visible Ear
% Simulator we can say for surtant that this is possible to simulate in
% real-time. But it is only possible because we do not need voxel models
% of the same magnitude as used in the ears simulator.

% \subsubsection{Breaking the tooth}
% As this sub simulator is the main topic of the thesis, it will be
% fulle described in the rest of the thesis. But to have an idear of
% what is to come this sub simulator models the tooth as a stiff
% material and tools for breaking the tooth are provided to the
% user. The model of the tooth uses triangles to model the surface
% and tetrahedrons for the body. We use the physics based elasticity
% theory and the finite element method to model deformation and
% energi. Lastly we employ fracture mechanics to model fracturing of the
% tooth when the energi raises above a limit defined for the material.

% \subsection{Removing pieces}
% This sub simulation is the least demanding simulation as all models
% in the sceen can be treated as ridig bodys, that is they cannot
% undergo deformation or other forms of modifications. They can only be
% move and rotated. So that main thing in this simulator is collision
% detection, which can be done naive on the GPU in real-time. The
% collision detection and tools needed here are very simulare to what is
% used in the heart simulation, and we are positive that this sub
% simulator will not be a problem to implement.

% \subsection{Stitching tissue}
% todo...

% \section{Binding the sub simulators together}
% Because the different sub simulator use different data models, we need
% to have a way of converting between the data models

% \subsection{Converting from Vertex/Primitive to Volumetric data}
% This convertion can be done by a algorithm known as flood filling.
% But to be able to use this algorithm the input model needs to be
% waterproof, meaning that the geometry must define a complete surface
% without any holes. A guide of how to do this convertion is provide in
% appendix \ref{guide:floodfilling}.

% \textbf{Generating Signed Distance Fields From Triangle Meshes}
% \url{http://citeseerx.ist.psu.edu/viewdoc/download?doi=10.1.1.111.6331&rep=rep1&type=pdf}

% \subsection{Converting from Volumetric to Vertex/Primitive data}
% This convertion can be done by a algorithm known as iso stuffing. ISO
% surface stuffing take a ISO surface as input, so to convert our
% density field we need to convert it to a ISO surface. A guide of how
% to do the density field to ISO surface converting and doing the ISO
% surface stuffing is provide in appendix \ref{guide:isosurfacestuffing}.

% \section{Hardware setup}
% Computer with monitor for visualisation, and some kind og haptic
% device for interacting with the simulator.

% \subsection{Haptic feedback}

% \begin{figure}[H]
%   \centering
%   \includegraphics[width=14cm]{./images/haptic_feedback.png}
%   \caption{Haptic feedback.}
%   \label{fig:haptic-feedback}
% \end{figure}

% \url{http://www.hjerteforeningen.dk/sw69890.asp}

%%% Local Variables: 
%%% mode: latex
%%% mode: auto-fill
%%% TeX-PDF-mode: t
%%% TeX-master: "../master.tex"
%%% End: 


%\subsection{Sean}
The process of moving an implicit surface given by a signed distance function is known as \emph{level set methods}. Level set methods are ways of influencing signed distance fields to move the implicit surfaces contained therein. This is done by solving certain equations of motion that we will describe in this section.

First we distinguish between Lagrangian and Eulerian representations of the surface.
To understand the difference between the two viewpoints we
imagine how we could measure the movement of ex. a fluid. In the Eulerian
viewpoint we would place measuring devices at fixed points in the
fluid, and continues sample the velocity of the fluid, the measured
value is taken as a avarage for an area. For convenience the
measuring devices are almost always placed in a uniform grid, with
square areas as illustrated in figure \ref{fig:eulerian}.

\begin{figure}[h]
  \centering
  \subfloat[Eulerian]{
    \includegraphics[width=0.5\textwidth]{imgs/eulerian.png}
    \label{fig:eulerian}
  }
  \subfloat[Lagrangian]{
    \includegraphics[width=0.5\textwidth]{imgs/lagrangian.png}
    \label{fig:lagrangian}
  }
  \caption{The eulerian and the lagrangian viewpoints.}
  \label{fig:eulerian-lagrangian}
\end{figure}

In the lagrangian viewpoint we let the measuring devices move along
the current in the fluid and take measurements at different locations
in the fluid. Here we imagine that the measuring device is part of the
fluid or a particle that the fluid moves around. Figure
\ref{fig:lagrangian} shows how the current has moved the devices as
time has progressed.

These two viewpoints are tightly coupled to the way the simulation data
is represented and visuliazed. Lets say that we are interested in
visualizing the surface of the substance.
%
The Lagrangian data is keept as points in space and moved around by
updating the position of the points. This can be visualized by
imposing that the points define a surface and rendered as geometry
ex. triangles between the points.
%
The eulerian viewpoint is seen as a 3D grid. Each cell in the grid has
a density that describes how much substance the cell contains (0-100\%).

In the Lagrangian representation, movement by a velocity field can be accomplished by solving the ordinary differential equation:
\begin{eqnarray}
\frac{d\vec{x}}{dt} = \vec{V}\left(\vec{x}\right)
\end{eqnarray}
As discussed previously, however, using implicit surfaces by an Eulerian representation rather than an explicit Lagrangian representation, such as a polygon mesh, provides certain benifits. Nowhere is this more clear than when implementing moving surfaces. Moving a surface built from triangles presents a number of problems. First and foremost, it becomes necessary to determine whether the surface starts to overlap itself and what to do if it does. Clearly this is problematic when dealing with polygons, since there is no obvious or "natural" way of merging two polygons which have overlapped.

\image{movingsurfaces.png}{0.4}{Two moving surface representations.
Top: Explicit representation, merging is a problem.
Bottom: Implicit surface by signed distance field, merging is automatic.}{velocity:movingsurfaces}

With implicit surfaces, this problem disappears, since an implicit surface is just that: implicit. If two "interior" areas of the implicit function "overlaps", it simply means that that area of the domain is in the interior of the function, and the interface will wrap around it as appropriate. See figure \vref{velocity:movingsurfaces} for an illustration of the advantage of implicit surfaces.

\subsection{The level set equation}
We examine now how to evolve an implicit surface, or interface, by affecting the underlying implicit function with an externally generated velocity field. This process of convection in an Eulerian representation is defined by equation \vref{velocity:levelseteq}
\begin{eqnarray}
  \label{velocity:levelseteq}
  \phi_t + \vec{V}\cdot \nabla \phi = 0
\end{eqnarray}
This partial differential equation is referred to as the \emph{level set equation} due to its central importance to level set methods. It describes the evolution of an implicit function $\phi$ by a velocity field $\vec{V}$. $\nabla \phi$, of course, is the gradient of the function. From this, we have
\begin{eqnarray}
\label{velocity:gradient}
\vec{V}\cdot\nabla\phi = u\phi_x + v\phi_y
\end{eqnarray}
Where $\phi_x$ and $\phi_y$ are the spacial derivatives in the first two
dimensions respectively and $u$ and $v$ are the two components of the
velocity vector.

Since we are essentially only interested in moving the implicit surface or interface with the velocity field, it is sufficient for the field to contain values only in a band around the interface. For simplicity of implementation, however, we assume that the field is defined across the entire domain.

For concrete implementation purposes, to evolve an implicit surface in an n-dimensional domain, the velocity field is an n-dimensional cartesian grid of n-dimensional vectors. In our two-dimensional case, that means all velocity fields are double arrays of two-value vectors.

\subsection{Upwind differencing}
How then, do we numerically solve equation \ref{velocity:levelseteq}?
As we know, the implicit function is discretized into a cartesian grid
of cells with $\Delta x$ representing the width and height of these
cells in the theoretical continuous field. In our case this is a
discretized signed distance field, each cell in the grid being of
course a number representing the distance from the
zero-isocountour. So too is the velocity field represented discretely
with vectors, as mentioned previously.

Now, since motion takes place across time, we will need discretize our
motion across time as well. We discretize time into steps of $\Delta
t$. The n'th time step we denote $t^n$ and the state of the cartesian
grid of our signed distance field at that time as $\phi^n$. Since the
velocity also may change over time, this too is separated into
discrete steps $\vec{V}^n$.

One way of discretizing equation \ref{velocity:levelseteq} is the simple \emph{forward Euler} method. Equation \vref{velocity:forwardeuler} shows how the time-dependant term $\phi_t$ of equation \ref{velocity:levelseteq} is discretized by forward Euler. 
\begin{eqnarray}
\label{velocity:forwardeuler}
\frac{\phi^{n+1}-\phi^n}{\Delta t}+\vec{V}^n\cdot\nabla\phi^n = 0
\end{eqnarray}
This is a first-order accurate method for time discretization, which \cit{osher2002level} suggests is adequate, based on practical experience.
Expanding the gradient as in \vref{velocity:gradient} we get
\begin{eqnarray}
\label{velocity:forwardeuler}
\frac{\phi^{n+1}-\phi^n}{\Delta t}+u\phi_{x}^n + v\phi_{y}^n = 0
\end{eqnarray}
To calculate this equation, then, we must find the spatial derivatives
in the $x$ and $y$ directions. For this we can again use a first-order
difference method. However, we need to pick a direction in which to
calculate the derivative. That is, do we use
\begin{eqnarray}
D^+\phi &\approx & \frac{\phi_{i+1} - \phi_i}{\Delta x} \texttt{ or}\\
D^-\phi &\approx & \frac{\phi_i - \phi_{i-1}}{\Delta x}
\end{eqnarray}
These being forward and backwards difference, respectively. Naturally, we choose by examining the velocities given for the cell in question in the velocity field and take the difference in the direction of change. Not surprisingly, the term "Upwind differencing" is derived from this way of sampling in the direction of change. 
Although it might seem natural to simply use a central difference, according to \cit{osher2002level}, this is unstable with forward Euler time discretization.

The method, then, for each cell in the grid of the implicit function is as follows:
\begin{itemize}
\item Look up the velocity in the corresponding cell in the velocity field.
\item Calculate the appropriate forward/backwards difference
\item From these, arrive at the partial spatial derivative
\item Store the new cell value
\end{itemize}
When this has been done for the entire grid, overwrite it with the new values.
Essentially, we are "collecting" the values that need to be written to the current cell of the grid in the direction they come from via the velocity grid.

To ensure stability of this method, \cit{osher2002level} recommends limiting the time-step according to the \emph{Courant-Friedrich-Lewy} condition (CFL for short) which can be written as
\begin{eqnarray}
\Delta t < \frac{\Delta x}{\max \left\lbrace \left| u \right| \right\rbrace}
\end{eqnarray}
By enforcing this, we increase the guarantee that small errors are not amplified over time. The effects of an unstable method are commonly referred to as "exploding", for obvious reasons.

Finally, \cit{osher2002level} recommends methods such as an essentially nonoscillatory (ENO) way of computing more accurate spatial differences and total variation diminishing (TVD) Runge-Kutta for further accuracy in temporal difference. In short, ENO is a polynomial forward or backward difference function and Runge-Kutta is essentially a multiple-step version of the simpler forward Euler method we use.

The above method, then, is the general solution to the problem of convecting an implicit surface by an externally generated velocity field. Using it, we may arbitrarily define our field of motion to produce motion in any way we wish.

%%% Local Variables: 
%%% mode: latex
%%% mode: auto-fill
%%% TeX-PDF-mode: t
%%% TeX-master: "../master.tex"
%%% End: 

\label{sec:extVel}

\section{Motion}


In contract to section \vref{sec:extVel}, this section describes
motion using a internally generated velocity field.

\newpage

\subsection{Grow/Shrink}

\todoPtx{Grow/shrink!}



%%% Local Variables: 
%%% mode: latex
%%% TeX-master: "../../master"
%%% End: 

% Copy this code to grow.tex later?

One of the most simple things we can do, is to move the interface in
the normal direction $\vec{N}$ using a constant $a$, efficiently
scaling the iso-surface.

From equation \vref{velocity:levelseteq} we have:
\begin{equation}
  \phi_t + \vec{V}\cdot \nabla \phi = 0
\end{equation}
Since $a\vec{N}$ have the same direction as $\nabla{\phi}$, we have the following:
\begin{align*}
  a\vec{N}\cdot\nabla\phi &=
  a\frac{\nabla\phi}{|\nabla\phi|}\cdot\nabla\phi \\
  &= a\frac{|\nabla\phi|^2}{|\nabla\phi|} 
  = a|\nabla\phi|
\end{align*}

With corresponds to this level set equation:

\begin{equation}
  \phi_t + a |\nabla \phi| = 0
\end{equation}

\todoPtx{When $\phi$ is a signed distance function, something, something....}

\todoPtx{Write about this code}
\begin{lstlisting}
for(unsigned int x=0; x<width; x++) {
    for(unsigned int y=0; y<height; y++) {
        phi(x,y) += -a;
    }
}
\end{lstlisting}

\subsection{Mean-Curvature}
\input{text/motion/curv}
\subsection{Morph}
\input{text/motion/morph}
\subsection{CFG condition (Stability)}
\input{text/motion/CFG}


%%% Local Variables: 
%%% mode: latex
%%% mode: auto-fill
%%% TeX-PDF-mode: t
%%% TeX-master: "../master.tex"
%%% End: 


\chapter{Extensions}
\label{chap:extensions}

\section{Narrow-band}
\label{sec:narrowband}
\todoVester{skriv om narrow band}

\section{3D}
\label{sec:3d}

\section{CUDA}
\label{sec:cuda}

Improving the performance of our algorithms can be done in many ways,
but one of the more obvious ones is using parallel computing!

Currently, the most accessible way to run programs in parallel is
using the graphics computation unit (GPU). Modern GPUs are very
powerful, and major manufactures have released software development
kits (SDK) for utilising the GPU for general purpose computation.

One of these SDKs is nVidias CUDA\cit{cuda}. The only thing needed to
use CUDA is a nvidia graphics card that is relatively new (a few years
tops), and the free SDK found at the CUDA website.

As GPUs were developed to render graphics, they are optimized to work
on spatially coherent data. This makes many of our algorithms a
natural target, as we often only need information about neighbouring
data points.

\section{Threads}

The CUDA programming model is centered about data parallel
programming. This means that you spawn a thread for each element in
your data, which runs the same program. In our case, this means
spawning a thread for each pixel. Luckily our algorithms are already
in this format, just with two \texttt{for}-loops iterating over the pixels.

Most of our algorithms uses this pattern:

\begin{lstlisting}
for (unsigned int i = 0; i<iterations; i++) {
    for (unsigned int x = 0; x < width; x++) {
        processPixel();
    }
}
\end{lstlisting}

Which is easily translated into:
\begin{lstlisting}
    const dim3 blockSize(32,16,1);
    const dim3 gridSize(width/blockSize.x, height/blockSize.y);
    processPixel<<gridSize,blockSize>>();
\end{lstlisting}

The grid and block size are telling CUDA how many threads to
spawn. A grid contains many blocks, and each block contain many
threads. In this sample, each block have $32 \times 16$ threads, and
the grid have $\frac{width}{32} \times \frac{height}{16}$ blocks. If
width and height are divisible with 32 and 16, this corresponds to
$width \times height$ threads. 

In CUDA 2.3, a block can contain no more than 512 threads, hence the
block size of $32\cdot 16 = 512$.

\todoPtx{More about threads?}

\section{Memory}

Memory in CUDA is divided in 3 parts. The Per-thread local memory than
only a single thread can access. Per-block shared memory which is
accessible to every thread in the same block and the global
memory that every thread can access. \todoPtx{Nice graphics?}

The difference between the shared and global memory is the speed. The
shared memory is much faster, but also much smaller (typically 16
KB). Its also inaccessible from threads in different blocks.

This make it challenging to utilise the whole GPU, as the algorithms
needs to be rethought.

One way to do this could be letting each thread fetch it's value
($\phi$ in our case) into the shared memory. Then if the threads are
organized in block where neighbouring threads are in the same block,
each thread can fetch the neighbouring pixels from the shared memory.

Such an optimization creates new challenges, as the threads near the
border cannot cross over to the next blocks shared memory. The
solution is to pad the area around the edges of the blocks, so if a
pixel is on the border, its run by both blocks. This makes us run a
few more threads than we have pixels, but it increases the speed as we
can exploit the shared memory.

A more simple type of optimization, is to use texture memory. GPUs
often need fast access to textures, so it have a texture cache
optimized for 2D spatial locality. This means that fetching data from
a texture will make fetching the neighbouring pixels faster.

\section{Implementation}

After a quick time profiling, we found that \texttt{Reinitialize} is
where our program spends most of its time, so this was the first to be
converted into CUDA.

The following is a very naive conversion. The code is almost the same
as the original in section \ref{sec:reinitialize}, and there are no
clever usage of shared memory or other optimizing tricks.

\begin{lstlisting}
#define GetPhi(phi,x,y,w) phi[x+w*(y)]

__global__ void reinit(float *phi,float* phi0, float* phin, 
                       unsigned int width, unsigned int height) {
    uint x = __umul24(blockIdx.x, blockDim.x) + threadIdx.x;
    uint y = __umul24(blockIdx.y, blockDim.y) + threadIdx.y;

    if (x > width || y > height)
        return;
    
    float xy = GetPhi(phi,x,y,width);

    float phiXPlus = 0.0f;
    float phiXMinus = 0.0f;
    float phiYPlus = 0.0f;
    float phiYMinus = 0.0f;        	
    if (x != width-1) phiXPlus  = (GetPhi(phi,x+1, y,width) - xy);
    if (x != 0)       phiXMinus = (xy - GetPhi(phi,x-1, y,width));
    
    if (y !=height-1) phiYPlus  = (GetPhi(phi,x, y+1,width) - xy);
    if (y != 0)       phiYMinus = (xy - GetPhi(phi,x, y-1,width));

    /* GetPhi(phin,x,y,width) = phiYPlus; */
    /* return; */


    float dXSquared = 0;
    float dYSquared = 0;
    float a = GetPhi(phi0,x,y,width);
    if (a > 0) {
        // formula 6.3 page 58
        float _max = max(phiXMinus, 0.0f);
        float _min = min(phiXPlus, 0.0f);
        dXSquared = max(_max*_max, _min*_min);
                    
        _max = max(phiYMinus, 0.0f);
        _min = min(phiYPlus, 0.0f);
        dYSquared = max(_max*_max, _min*_min);
    } else {
        // formula 6.4 page 58
        float _max = max(phiXPlus, 0.0f);
        float _min = min(phiXMinus, 0.0f);
        dXSquared = max(_max*_max, _min*_min);
                    
        _max = max(phiYPlus, 0.0f);
        _min = min(phiYMinus, 0.0f);
        dYSquared = max(_max*_max, _min*_min);        				
    }

    float normSquared = dXSquared + dYSquared;           
    float norm = sqrt(normSquared);

    // Using the S(phi) sign formula 7.6 on page 67
    //float sign = phi(x,y) / sqrt(phi(x,y)*phi(x,y) + normSquared);
    float sign = GetPhi(phi0,x,y,width) / 
        sqrt(GetPhi(phi0,x,y,width)*GetPhi(phi0,x,y,width) + 1);
    float t = 0.3; // A stabil CFL condition
    GetPhi(phin,x,y,width) = GetPhi(phi,x,y,width) - sign*(norm - 1)*t;


}
\end{lstlisting}

Coping the data, and starting the threads are done in the following code:

\begin{lstlisting}
void cu_Reinit(float* data, 
               unsigned int w,
               unsigned int h,
               unsigned int iterations) {
    float* phiData;
    float* phi0Data;
    float* phinData;

    cudaMalloc((void**)&phiData, sizeof(float)*w*h);
    cudaMalloc((void**)&phi0Data, sizeof(float)*w*h);
    cudaMalloc((void**)&phinData, sizeof(float)*w*h);
    cudaMemcpy((void*)phiData,(void*)data, sizeof(float)*w*h,
               cudaMemcpyHostToDevice);
    cudaMemcpy((void*)phi0Data,(void*)data, sizeof(float)*w*h,
               cudaMemcpyHostToDevice);
    cudaMemcpy((void*)phinData,(void*)data, sizeof(float)*w*h,
               cudaMemcpyHostToDevice);


    CHECK_FOR_CUDA_ERROR();

    const dim3 blockSize(32,16,1);
    const dim3 gridSize(w/blockSize.x, h/blockSize.y);

    for (unsigned int i=0;i<iterations;i++) {
        reinit<<<gridSize,blockSize>>>(phiData,phi0Data,phinData,w,h);
        float* tmp = phiData;
        phiData = phinData;
        phinData = tmp;

        cudaThreadSynchronize();
        CHECK_FOR_CUDA_ERROR();
    }

    cudaMemcpy((void*)data,(void*)phiData,
                sizeof(float)*w*h,cudaMemcpyDeviceToHost);
    CHECK_FOR_CUDA_ERROR();
    cudaFree(phiData);
    cudaFree(phi0Data);
    cudaFree(phinData);
}

\end{lstlisting}

\section{Results \& Conclusion}

The results in table \ref{tbl:cudaRes} are taken from a system with a
1.8 Ghz Intel Core 2 Duo CPU, 4 GB RAM and a 512MB nVidia GeForce
9600M GT. The time is an average of about 100 iterations of the
algorithm.

\begin{table}[h]
  \centering
  \begin{tabular}{|l|r|r|r|}
    \hline    Algorithm & CPU & GPU & Speedup \\
    % BEGIN RECEIVE ORGTBL numbers
\hline
Reinitialization & 417825 usec & 136675 usec & 3.0570697 \\
- with textures & - & 100006 usec & 4.1779993 \\
\hline
    % END RECEIVE ORGTBL numbers
  \end{tabular}
  \caption{GPU vs. CPU comparison}
  \label{tbl:cudaRes}
\end{table}

% grep Reinit run1.log | grep CUDA | awk '{sum+=$7} END {print "avg=",sum/NR}'

\begin{comment}
#+ORGTBL: SEND numbers orgtbl-to-latex :splice t :skip 2
|------------------+-------------+-------------+-----------|
|                  | CPU         | GPU         |   Speedup |
|------------------+-------------+-------------+-----------|
| Reinitialization | 417825 usec | 136675 usec | 3.0570697 |
| - with textures  | -           | 100006 usec | 4.1779993 |
|------------------+-------------+-------------+-----------|
#+TBLFM: @2$4=@2$2 / @2$3::@3$4=@2$2 / @3$3
\end{comment}

The results shows a significant speedup. Using a quite naive
implementation the speedup is easily tripled on a inexpensive consumer
graphics card.

Using textures to cache lookup, we gain even more performance, going
from 3x to 4x. If we'd had more time more optimization techniques
could have been applied. E.g. using shared memory which most likely
would have improved performance even more.


\todoPtx{Picture!}




%%% Local Variables: 
%%% mode: latex
%%% mode: auto-fill
%%% mode: orgtbl
%%% TeX-PDF-mode: t
%%% TeX-master: "../master.tex"
%%% End: 


\section{Segmentation}
\label{sec:segmentation}
% \image{image, scale, caption, label}
\image{segmentation.png}{0.3}{A segmentation in progress.}{segmentation:fig:intro}

Segmentation is an incredibly important area of interest when it comes to Medical Imaging. Segmentation is the problem of partitioning a digital image into multiple segments that is more meaningful and easier to analyse. Typically one would like to locate the boundaries in a picture such as lines, curves, etc.

The result of a segmentation is a set og segments that covers the entire picture. All pixels in each segment shares properties based one how the picture is segmented. It could be color or intensity. Adjecent regions are significantly different based on these characteristics.

A technique is to initially start inside the object you want to segment and then expand it like a balloon until the surface reaches the edge of the contour.

To illustrate segmentation in a level set model, I have implemented two different algorithms which are described in sections \ref{segmentation:sec:algorithm1} and \ref{segmentation:sec:algorithm2}.

\subsection{Implicit vs. Explicit representation}

Since segmentation techniques normally are used to locate organs in MR scans or meassure the volume of tissue, eg. from real people, it is very important that the segmentation is correct and that it is fast. Therefore, we have to convinse our selves that our technique can find the contour in images even though they can contain a lot of noise and artefacts. 

In an explicit representation, we have the problem that when we only represent the surface, we run into trouble when segmenting artefacts as can be seen in figure \ref{segmentation:fig:explicit}. The problem is that the segmentation can not figure out to skip over the artefacts which resolves in a segmentation that never terminates. There exist algorithms that try to skip the artefacts, but they are prone to failure.

% \image{image, scale, caption, label}
\image{explicit.png}{0.3}{We see how the segmentation, using an explicit representation has trouble with artefacts in the picture. The surface is about to wrap around it self creating an endless loop around the artefact.}{segmentation:fig:explicit}

In an implicit representation, the problem vanishes since we look at the larger picture and not just the boundary of the current segmentation. Because we use a level set to solve the problem, when our algorithm reaches an artefact the solver simply goes around it at merge at the other side.

\subsection{Algorithm 1 - Moving in the normal direction}
\label{segmentation:sec:algorithm1}

In this algorithm I have been inspired by the balloon algorithm. 
To segment a part of an image, we start with a small area inside the area we want to segment and grow it in the normal direction if we have not reached the boundary yet. We know if we have hit the boundary if the value of the pixel is smaller/larger than a specified treshold we define. If we have crossed the border (again based on the threshold value), we shrink that point of the surface by going in the reverse direction of the normal.

% \image{image, scale, caption, label}
\image{normalDirection.png}{0.3}{In the figure, we see how the segmentation grows in the normal direction}{segmentation:fig:normaldirection}


Basically we solve the following equation:

\begin{equation}
  \phi_{t} + a|\nabla{\phi}| = 0
\end{equation}

which is easily achieved using the code below: 

\begin{lstlisting}[language=c++]
\label{segmentation:code}
for(unsigned int x=0; x<width; x++) {
    for(unsigned int y=0; y<height; y++) {
        [...]
        if (picture(x,y) > threshold) {
            phi(x,y) += a;
            growth += -a;
        } else {
            phi(x,y) += -a;
            growth += a;
        }
    }
}
[...]

if (growth / ``number of pixels moving'' < tTreshold) {
    done = true;
}
\end{lstlisting}



We want to be able to terminate the segmentation when we have found the correct area. We do this by looking at the zero iso-surface and check how much the surface is moving. When it slows down we know that we have found the area we want to segment.

This is a quite simple algorithm which surpricingly produces good results. On a picture of dimensions 512 x 512 it stops after approximately 250-300 iterations which is quite good. We could do this many times faster if we choose to implement the reinitialization step on the GPU via CUDA.

In order to make the segmentation algorithm stop when it reaches the boundary, we calculate the following factor:

\begin{equation*}
 \textrm{factor} = \textrm{growth } / \textrm{ number of pixels moving on iso-surface}
\end{equation*}

Which is a number that goes towards zero when the iso surface stops moving. To see this think about what happens when we have reached the boundary. since about half of the iso-surface is increased and the other half is going to be decreased the factor should be approximately zero. tThreshold is set to 0.03 though experiments.


\subsection{Algortihm 2 - Edge detection}
\label{segmentation:sec:algorithm2}

% \image{image, scale, caption, label}
\image{placeholder.png}{0.3}{Some text}{segmentation:fig:advancedsegmentation}

Algorithm 2 is more advanced and tries to find the edges beforehand to increase the likelyhood that we segment the correct part. It builds a series of images and solves the following equation:

\begin{equation}
\label{segmentation:equation:advanced}
  \dfrac{\partial \phi}{\partial t} - \textrm{grad}(D) \cdot \textrm{grad}(\phi) = 0
\end{equation}

where image D has the edge information with an edge denoted as a one and a nonedge as a zero.

To compute image D we have to go through a number of steps. First, we compute an image A where every pixel is the norm of the gradient in the original image. Secondly, we compute an image B where every pixel is the gradient in image A dotted with the normal in the original image. Compute an image C where every pixel is the absolute value of the gradient in the original image dotted with the normal.
With this information, it is now possible to calculate the zero crossings. A zero crossing is defined to be either that one of the neighbouring pixels have a different sign than the current pixel, or that with the value of the current pixel is zero. And finally, if the corresponding value in image C is larger than some specified value then it is also a zero crossing. The final image, D, is calcutated by setting all pixels with an edge to one and all non-edges to zero. We need to run the reinitialization method on the image to make sure that all distances are correct. In this particulary instance, we need to do a thousand iterations. These calculations are all computed as a preprocessing step before we iteratively solve the level set equation (\ref{segmentation:equation:advanced}).


To solve the level set equation we implement the following code:
\begin{lstlisting}[language=c++]
for(unsigned int x=0; x<width; x++) {
    for(unsigned int y=0; y<height; y++) {
        phi(x,y) +=  (gradD(x,y) * gradPhi(x,y)) * time;
    }
}
\end{lstlisting}

Where time is the factor:

\begin{equation*}
  \dfrac{\Delta x} {\max \{|\textrm{gradient}(x,y)|\}} 
\end{equation*}

When solving the level set equation, for every pixel we get a vector that goes away from the iso-surface and points at the closest edge in the normal direction.

\subsection{Conclusion}
\label{segmentation:conclusion}
%% Wrap up.

I have implemented two algorithms for segmenting pictures where the first is a simple algorithm that grows only based on the normal of the iso-surface and stops its segmentation when the iso-surface encounters pixelvalues that cross the threshold specified.

The second algorithm is more advanced and makes good use of the information from the gradient by growing in the direction of the edges and the normal.

Do to time constraints I have only tested the algorithms on grayscale pictures and also not on real medical data, but none the less I still get results that should scale to real data.



\section{Fluid / Smoke}
\label{sec:fluid}

\section{Image reconstruction}
\label{sec:imagereconstruction}

\appendix

\newpage


\bibliographystyle{alpha}
\bibliography{text/levelset}


\end{document}


%%% Local Variables: 
%%% mode: latex
%%% mode: auto-fill
%%% TeX-PDF-mode: t
%%% TeX-master: t
%%% End: 
