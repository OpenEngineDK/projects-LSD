\documentclass[a4paper, oneside]{memoir}
\usepackage[utf8]{inputenc}
\usepackage{graphicx}
\usepackage{listings}
\usepackage{amsthm}
\usepackage{amsfonts}
\usepackage{amsmath}
\usepackage{varioref}
\usepackage{color}
\usepackage{pgf}
\usepackage[colorinlistoftodos]{todonotes}
%\usepackage[colorlinks, linkcolor=blue, citecolor=red, urlcolor=brown]{hyperref}
\usepackage{hyperref}
% ^- farver gør det meget nederen at udskrive en s/h kopi, da alle "se
% fig 1.1 på side xx" bliver grå! (ptx)
% Individual todos
\newcommand{\todoPtx}[2][]{\todo[color=red,    #1]{Ptx: #2}}
\newcommand{\todoCpvc}[2][]{\todo[color=yellow, #1]{Cpvc: #2}}
\newcommand{\todoSean}[2][]{\todo[color=pink,   #1]{Sean: #2}}
\newcommand{\todoHave}[2][]{\todo[color=green,  #1]{Have: #2}}
\newcommand{\todoVester}[2][]{\todo[color=blue,  #1]{Vester: #2}}

\synctex=1


% Dokumantation:
% http://www.tex.ac.uk/tex-archive/macros/latex/contrib/todonotes/todonotes.pdf
% Vigtigste kommandoer:
% \todoVester[inline]{text}
% \missingfigure{A illustration of how peers are placed on the ring}
% \todo{Introduction to this section}
% \todo[inline]{Someone write this}

% replaces cite. Use \cit instead!
\newcommand{\cit}[1] {
  \cite{#1}
}

% \image{image, scale, caption, label}
\newcommand{\image}[4]{
  \begin{figure*}[!htb]
    \centering
    \includegraphics[scale=#2]{imgs/#1}
    \caption{#3}
    \label{#4}
  \end{figure*}}
% ^- Det her går aldrig godt, man har aldrig brug for at ændre flere
% ting end der er muligt med denne slags makroer. (ptx)

\title{Level Sets!!}
\author{
 \\
  Martin Have (2005????)\\
  \url{have@cs.au.dk}\\
  \\
  Peter Kristensen (20051866)\\
  \url{ptx@cs.au.dk}\\
  \\
  Mikkel Vester (20053229)\\
  \url{vester@cs.au.dk}\\
  \\
  Cpvc (200?????)\\
  \url{cpvc@cs.au.dk}\\
  \\
  Hitler (Moralsk support)\\
  \url{hitler@cs.au.dk}\\
  \\
  University of Aarhus
}

\begin{document}

\maketitle{}


\missingfigure{Her er et billede af et morph ting}

% Hvem laver hvad.
\begin{comment} % orgtbl-mode er win
|----------------------------+----------+--------|
| Afsnit                     | Indhold  | Hvem   |
|----------------------------+----------+--------|
| Intro                      | ch. 1,2  | have   |
| Build SDF / Discretization | artikel? | vester |
| Reinitialize               | ch. 7    | cpvc   |
| Motion                     | ch. 4,6  | ptx    |
| Externally gen. vf.        | ch. 3    | geggie |
| Godunov                    | ch. 5    | cpvc   |
|----------------------------+----------+--------|
\end{comment}

\newpage

\tableofcontents{}
\listoftodos
\chapter{Introduction}
\label{chap:introduction}
  When simulating physical phenomena one must choose a model which fits
the phenomena being modeled. The level set method (LSM), is a method
for modelling phenomena that can be described in terms of moving level
sets.
%
A level set in itself is only a mathematical function that groups
variables which have the same function value. It describes a
parametric function in one dimension higher than its original domain,
gaining the ability to describe more than one not colocated level set
in the same structure.
%
A two dimensional level set is known as a level curve or isocontour,
which uses two dimensions to describe curves. We encounter isocontours
in everyday life as pressure information in the weather forecast and
terrain elevation on maps.
%
A three dimensional level set is
known as a level surface or isosurface, and is used, as suggested be
it's name, to describe surfaces.  A general a level set in n
dimensions is used to describe n-1 dimensional objects. Mathematically
a level set is define as follows:

\begin{equation}
\{ (x_1,...,x_n) \} | f(x_1,...,x_n) = c
\end{equation}

In words this means that all values of $v_i$ leading to a function
value of $c$ defines the level set $c$.
%
The level set in its own right does not make a simulation. Unless it
is evolved over times it stays the same. To evolve the level set over
time, partial differential equations (PDEs) describing the physical
phenomena needs to be solved to move the level sets.
%
Typical phenomena modeled by the LSM includes: computational fluid
dynamics (CFG), and fire and explosion simulation. 

In this report, we will explain what the Level Set method is and what
its applications are. Furthermore, we will provide code examples of
how to implement the mathematical formulas.

In figure \vref{fig:heightmap} we see two figures describing a two dimensional and
three dimensional view of an island. Figure \vref{fig:isocontour-2d} describes an
isocontour map of heights where the color indicates the height of each
point. The brighter the color the higher the point. Figure
\vref{fig:isocontour-3d} shows the corresponding 3d view.

\begin{figure}[h]
\begin{center}
  \subfloat[Two dimensional view of an isocontour map]{
    \includegraphics[width=0.5\textwidth]{imgs/226171_226171.png}
    \label{fig:isocontour-2d}
  }
  \subfloat[Three dimensional view of figure \ref{fig:isocontour-2d}]{
    \includegraphics[width=0.5\textwidth]{imgs/226170_226170.png}
    \label{fig:isocontour-3d}
  }
\end{center}
\caption{Illustrations of heightmap, from: Intel Array Visualizer
  Gallery, \cit{intel}}
\label{fig:heightmap}
\end{figure}


For a level set representation there is an underlying function
describing the topology. Each contour corresponds to a set of points
in this function with the same value, thus we need a function which
produces a field of values. In heightmaps, it would be a function from
$(x,y)$ coordinates to a height. In general, any function producing a
field of values is sufficient.  

One function that has the desired characteristics is a function known
as the signed distance function $\phi$.


\section*{Outline of the report}

In chapter \vref{chap:sdf}, we give an in depth look on the signed
distance function, describe what kinds of mathematical operations we
have in our toolbox and describe the important reinitialize function,
section \vref{sec:reinitialize}.

We have individually made a number of extensions to the basic level
set implementation. These extensions should show the versatility of
the method and show some practical applications.

In chapter \vref{chap:extensions}, we look at these extensions. Each
section has been written by the person responsible for it.
In section \vref{sec:narrowband}, for efficiency, we improve the
signed distance implementation to use the narrow band technique. 
In section \vref{sec:cuda} we improve the reinitialization step by
implementing it on the GPU.
In section \vref{sec:segmentation}, we look at how to implement
segmentation algorithms which can be used in mediacal imaging. 
In section \vref{sec:fluid}, we implement a fluid solver for computer
graphics using the level set method. 



%%% Local Variables: 
%%% mode: latex 
%%% mode: auto-fill 
%%% TeX-PDF-mode: t 
%%% TeX-master: "../master.tex" 
%%% End:



\chapter{SDF}
\label{chap:sdf}
\todoVester{tekst her}

\section{Mikkel-Sean Algorithm TM}
\todo{Dette er et eksempel på en todo for alle}

\section{Reinitialize}
\label{sec:reinitialize}

\section{CSG (Union, Intersetion, Minus)}
\subsection{Union}
\subsection{Intersetion}
\subsection{Minus}

\section{Motion (grow/shrink, mean-curvature, morph, CFG condition)}
\subsection{Grow/Shrink}

\todoPtx{Grow/shrink!}



%%% Local Variables: 
%%% mode: latex
%%% TeX-master: "../../master"
%%% End: 

\subsection{Mean-Curvature}
\input{text/motion/curv}
\subsection{Morph}
\input{text/motion/morph}
\subsection{CFG condition (Stability)}
\input{text/motion/CFG}

\chapter{Extensions}
\label{chap:extensions}

\section{Narrow-band}
\label{sec:narrowband}
\todoVester{skriv om narrow band}

\section{3D}
\label{sec:3d}

\section{CUDA}
\label{sec:cuda}

Improving the performance of our algorithms can be done in many ways,
but one of the more obvious ones is using parallel computing!

Currently, the most accessible way to run programs in parallel is
using the graphics computation unit (GPU). Modern GPUs are very
powerful, and major manufactures have released software development
kits (SDK) for utilising the GPU for general purpose computation.

One of these SDKs is nVidias CUDA\cit{cuda}. The only thing needed to
use CUDA is a nvidia graphics card that is relatively new (a few years
tops), and the free SDK found at the CUDA website.

As GPUs were developed to render graphics, they are optimized to work
on spatially coherent data. This makes many of our algorithms a
natural target, as we often only need information about neighbouring
data points.

\section{Threads}

The CUDA programming model is centered about data parallel
programming. This means that you spawn a thread for each element in
your data, which runs the same program. In our case, this means
spawning a thread for each pixel. Luckily our algorithms are already
in this format, just with two \texttt{for}-loops iterating over the pixels.

Most of our algorithms uses this pattern:

\begin{lstlisting}
for (unsigned int i = 0; i<iterations; i++) {
    for (unsigned int x = 0; x < width; x++) {
        processPixel();
    }
}
\end{lstlisting}

Which is easily translated into:
\begin{lstlisting}
    const dim3 blockSize(32,16,1);
    const dim3 gridSize(width/blockSize.x, height/blockSize.y);
    processPixel<<gridSize,blockSize>>();
\end{lstlisting}

The grid and block size are telling CUDA how many threads to
spawn. A grid contains many blocks, and each block contain many
threads. In this sample, each block have $32 \times 16$ threads, and
the grid have $\frac{width}{32} \times \frac{height}{16}$ blocks. If
width and height are divisible with 32 and 16, this corresponds to
$width \times height$ threads. 

In CUDA 2.3, a block can contain no more than 512 threads, hence the
block size of $32\cdot 16 = 512$.

\todoPtx{More about threads?}

\section{Memory}

Memory in CUDA is divided in 3 parts. The Per-thread local memory than
only a single thread can access. Per-block shared memory which is
accessible to every thread in the same block and the global
memory that every thread can access. \todoPtx{Nice graphics?}

The difference between the shared and global memory is the speed. The
shared memory is much faster, but also much smaller (typically 16
KB). Its also inaccessible from threads in different blocks.

This make it challenging to utilise the whole GPU, as the algorithms
needs to be rethought.

One way to do this could be letting each thread fetch it's value
($\phi$ in our case) into the shared memory. Then if the threads are
organized in block where neighbouring threads are in the same block,
each thread can fetch the neighbouring pixels from the shared memory.

Such an optimization creates new challenges, as the threads near the
border cannot cross over to the next blocks shared memory. The
solution is to pad the area around the edges of the blocks, so if a
pixel is on the border, its run by both blocks. This makes us run a
few more threads than we have pixels, but it increases the speed as we
can exploit the shared memory.

A more simple type of optimization, is to use texture memory. GPUs
often need fast access to textures, so it have a texture cache
optimized for 2D spatial locality. This means that fetching data from
a texture will make fetching the neighbouring pixels faster.

\section{Implementation}

After a quick time profiling, we found that \texttt{Reinitialize} is
where our program spends most of its time, so this was the first to be
converted into CUDA.

The following is a very naive conversion. The code is almost the same
as the original in section \ref{sec:reinitialize}, and there are no
clever usage of shared memory or other optimizing tricks.

\begin{lstlisting}
#define GetPhi(phi,x,y,w) phi[x+w*(y)]

__global__ void reinit(float *phi,float* phi0, float* phin, 
                       unsigned int width, unsigned int height) {
    uint x = __umul24(blockIdx.x, blockDim.x) + threadIdx.x;
    uint y = __umul24(blockIdx.y, blockDim.y) + threadIdx.y;

    if (x > width || y > height)
        return;
    
    float xy = GetPhi(phi,x,y,width);

    float phiXPlus = 0.0f;
    float phiXMinus = 0.0f;
    float phiYPlus = 0.0f;
    float phiYMinus = 0.0f;        	
    if (x != width-1) phiXPlus  = (GetPhi(phi,x+1, y,width) - xy);
    if (x != 0)       phiXMinus = (xy - GetPhi(phi,x-1, y,width));
    
    if (y !=height-1) phiYPlus  = (GetPhi(phi,x, y+1,width) - xy);
    if (y != 0)       phiYMinus = (xy - GetPhi(phi,x, y-1,width));

    /* GetPhi(phin,x,y,width) = phiYPlus; */
    /* return; */


    float dXSquared = 0;
    float dYSquared = 0;
    float a = GetPhi(phi0,x,y,width);
    if (a > 0) {
        // formula 6.3 page 58
        float _max = max(phiXMinus, 0.0f);
        float _min = min(phiXPlus, 0.0f);
        dXSquared = max(_max*_max, _min*_min);
                    
        _max = max(phiYMinus, 0.0f);
        _min = min(phiYPlus, 0.0f);
        dYSquared = max(_max*_max, _min*_min);
    } else {
        // formula 6.4 page 58
        float _max = max(phiXPlus, 0.0f);
        float _min = min(phiXMinus, 0.0f);
        dXSquared = max(_max*_max, _min*_min);
                    
        _max = max(phiYPlus, 0.0f);
        _min = min(phiYMinus, 0.0f);
        dYSquared = max(_max*_max, _min*_min);        				
    }

    float normSquared = dXSquared + dYSquared;           
    float norm = sqrt(normSquared);

    // Using the S(phi) sign formula 7.6 on page 67
    //float sign = phi(x,y) / sqrt(phi(x,y)*phi(x,y) + normSquared);
    float sign = GetPhi(phi0,x,y,width) / 
        sqrt(GetPhi(phi0,x,y,width)*GetPhi(phi0,x,y,width) + 1);
    float t = 0.3; // A stabil CFL condition
    GetPhi(phin,x,y,width) = GetPhi(phi,x,y,width) - sign*(norm - 1)*t;


}
\end{lstlisting}

Coping the data, and starting the threads are done in the following code:

\begin{lstlisting}
void cu_Reinit(float* data, 
               unsigned int w,
               unsigned int h,
               unsigned int iterations) {
    float* phiData;
    float* phi0Data;
    float* phinData;

    cudaMalloc((void**)&phiData, sizeof(float)*w*h);
    cudaMalloc((void**)&phi0Data, sizeof(float)*w*h);
    cudaMalloc((void**)&phinData, sizeof(float)*w*h);
    cudaMemcpy((void*)phiData,(void*)data, sizeof(float)*w*h,
               cudaMemcpyHostToDevice);
    cudaMemcpy((void*)phi0Data,(void*)data, sizeof(float)*w*h,
               cudaMemcpyHostToDevice);
    cudaMemcpy((void*)phinData,(void*)data, sizeof(float)*w*h,
               cudaMemcpyHostToDevice);


    CHECK_FOR_CUDA_ERROR();

    const dim3 blockSize(32,16,1);
    const dim3 gridSize(w/blockSize.x, h/blockSize.y);

    for (unsigned int i=0;i<iterations;i++) {
        reinit<<<gridSize,blockSize>>>(phiData,phi0Data,phinData,w,h);
        float* tmp = phiData;
        phiData = phinData;
        phinData = tmp;

        cudaThreadSynchronize();
        CHECK_FOR_CUDA_ERROR();
    }

    cudaMemcpy((void*)data,(void*)phiData,
                sizeof(float)*w*h,cudaMemcpyDeviceToHost);
    CHECK_FOR_CUDA_ERROR();
    cudaFree(phiData);
    cudaFree(phi0Data);
    cudaFree(phinData);
}

\end{lstlisting}

\section{Results \& Conclusion}

The results in table \ref{tbl:cudaRes} are taken from a system with a
1.8 Ghz Intel Core 2 Duo CPU, 4 GB RAM and a 512MB nVidia GeForce
9600M GT. The time is an average of about 100 iterations of the
algorithm.

\begin{table}[h]
  \centering
  \begin{tabular}{|l|r|r|r|}
    \hline    Algorithm & CPU & GPU & Speedup \\
    % BEGIN RECEIVE ORGTBL numbers
\hline
Reinitialization & 417825 usec & 136675 usec & 3.0570697 \\
- with textures & - & 100006 usec & 4.1779993 \\
\hline
    % END RECEIVE ORGTBL numbers
  \end{tabular}
  \caption{GPU vs. CPU comparison}
  \label{tbl:cudaRes}
\end{table}

% grep Reinit run1.log | grep CUDA | awk '{sum+=$7} END {print "avg=",sum/NR}'

\begin{comment}
#+ORGTBL: SEND numbers orgtbl-to-latex :splice t :skip 2
|------------------+-------------+-------------+-----------|
|                  | CPU         | GPU         |   Speedup |
|------------------+-------------+-------------+-----------|
| Reinitialization | 417825 usec | 136675 usec | 3.0570697 |
| - with textures  | -           | 100006 usec | 4.1779993 |
|------------------+-------------+-------------+-----------|
#+TBLFM: @2$4=@2$2 / @2$3::@3$4=@2$2 / @3$3
\end{comment}

The results shows a significant speedup. Using a quite naive
implementation the speedup is easily tripled on a inexpensive consumer
graphics card.

Using textures to cache lookup, we gain even more performance, going
from 3x to 4x. If we'd had more time more optimization techniques
could have been applied. E.g. using shared memory which most likely
would have improved performance even more.


\todoPtx{Picture!}




%%% Local Variables: 
%%% mode: latex
%%% mode: auto-fill
%%% mode: orgtbl
%%% TeX-PDF-mode: t
%%% TeX-master: "../master.tex"
%%% End: 


\section{Segmentation}
\label{sec:segmentation}
  % \image{image, scale, caption, label}
\image{segmentation.png}{0.3}{A segmentation in progress.}{segmentation:fig:intro}

Segmentation is an incredibly important area of interest when it comes to Medical Imaging. Segmentation is the problem of partitioning a digital image into multiple segments that is more meaningful and easier to analyse. Typically one would like to locate the boundaries in a picture such as lines, curves, etc.

The result of a segmentation is a set og segments that covers the entire picture. All pixels in each segment shares properties based one how the picture is segmented. It could be color or intensity. Adjecent regions are significantly different based on these characteristics.

A technique is to initially start inside the object you want to segment and then expand it like a balloon until the surface reaches the edge of the contour.

To illustrate segmentation in a level set model, I have implemented two different algorithms which are described in sections \ref{segmentation:sec:algorithm1} and \ref{segmentation:sec:algorithm2}.

\subsection{Implicit vs. Explicit representation}

Since segmentation techniques normally are used to locate organs in MR scans or meassure the volume of tissue, eg. from real people, it is very important that the segmentation is correct and that it is fast. Therefore, we have to convinse our selves that our technique can find the contour in images even though they can contain a lot of noise and artefacts. 

In an explicit representation, we have the problem that when we only represent the surface, we run into trouble when segmenting artefacts as can be seen in figure \ref{segmentation:fig:explicit}. The problem is that the segmentation can not figure out to skip over the artefacts which resolves in a segmentation that never terminates. There exist algorithms that try to skip the artefacts, but they are prone to failure.

% \image{image, scale, caption, label}
\image{explicit.png}{0.3}{We see how the segmentation, using an explicit representation has trouble with artefacts in the picture. The surface is about to wrap around it self creating an endless loop around the artefact.}{segmentation:fig:explicit}

In an implicit representation, the problem vanishes since we look at the larger picture and not just the boundary of the current segmentation. Because we use a level set to solve the problem, when our algorithm reaches an artefact the solver simply goes around it at merge at the other side.

\subsection{Algorithm 1 - Moving in the normal direction}
\label{segmentation:sec:algorithm1}

In this algorithm I have been inspired by the balloon algorithm. 
To segment a part of an image, we start with a small area inside the area we want to segment and grow it in the normal direction if we have not reached the boundary yet. We know if we have hit the boundary if the value of the pixel is smaller/larger than a specified treshold we define. If we have crossed the border (again based on the threshold value), we shrink that point of the surface by going in the reverse direction of the normal.

% \image{image, scale, caption, label}
\image{normalDirection.png}{0.3}{In the figure, we see how the segmentation grows in the normal direction}{segmentation:fig:normaldirection}


Basically we solve the following equation:

\begin{equation}
  \phi_{t} + a|\nabla{\phi}| = 0
\end{equation}

which is easily achieved using the code below: 

\begin{lstlisting}[language=c++]
\label{segmentation:code}
for(unsigned int x=0; x<width; x++) {
    for(unsigned int y=0; y<height; y++) {
        [...]
        if (picture(x,y) > threshold) {
            phi(x,y) += a;
            growth += -a;
        } else {
            phi(x,y) += -a;
            growth += a;
        }
    }
}
[...]

if (growth / ``number of pixels moving'' < tTreshold) {
    done = true;
}
\end{lstlisting}



We want to be able to terminate the segmentation when we have found the correct area. We do this by looking at the zero iso-surface and check how much the surface is moving. When it slows down we know that we have found the area we want to segment.

This is a quite simple algorithm which surpricingly produces good results. On a picture of dimensions 512 x 512 it stops after approximately 250-300 iterations which is quite good. We could do this many times faster if we choose to implement the reinitialization step on the GPU via CUDA.

In order to make the segmentation algorithm stop when it reaches the boundary, we calculate the following factor:

\begin{equation*}
 \textrm{factor} = \textrm{growth } / \textrm{ number of pixels moving on iso-surface}
\end{equation*}

Which is a number that goes towards zero when the iso surface stops moving. To see this think about what happens when we have reached the boundary. since about half of the iso-surface is increased and the other half is going to be decreased the factor should be approximately zero. tThreshold is set to 0.03 though experiments.


\subsection{Algortihm 2 - Edge detection}
\label{segmentation:sec:algorithm2}

% \image{image, scale, caption, label}
\image{placeholder.png}{0.3}{Some text}{segmentation:fig:advancedsegmentation}

Algorithm 2 is more advanced and tries to find the edges beforehand to increase the likelyhood that we segment the correct part. It builds a series of images and solves the following equation:

\begin{equation}
\label{segmentation:equation:advanced}
  \dfrac{\partial \phi}{\partial t} - \textrm{grad}(D) \cdot \textrm{grad}(\phi) = 0
\end{equation}

where image D has the edge information with an edge denoted as a one and a nonedge as a zero.

To compute image D we have to go through a number of steps. First, we compute an image A where every pixel is the norm of the gradient in the original image. Secondly, we compute an image B where every pixel is the gradient in image A dotted with the normal in the original image. Compute an image C where every pixel is the absolute value of the gradient in the original image dotted with the normal.
With this information, it is now possible to calculate the zero crossings. A zero crossing is defined to be either that one of the neighbouring pixels have a different sign than the current pixel, or that with the value of the current pixel is zero. And finally, if the corresponding value in image C is larger than some specified value then it is also a zero crossing. The final image, D, is calcutated by setting all pixels with an edge to one and all non-edges to zero. We need to run the reinitialization method on the image to make sure that all distances are correct. In this particulary instance, we need to do a thousand iterations. These calculations are all computed as a preprocessing step before we iteratively solve the level set equation (\ref{segmentation:equation:advanced}).


To solve the level set equation we implement the following code:
\begin{lstlisting}[language=c++]
for(unsigned int x=0; x<width; x++) {
    for(unsigned int y=0; y<height; y++) {
        phi(x,y) +=  (gradD(x,y) * gradPhi(x,y)) * time;
    }
}
\end{lstlisting}

Where time is the factor:

\begin{equation*}
  \dfrac{\Delta x} {\max \{|\textrm{gradient}(x,y)|\}} 
\end{equation*}

When solving the level set equation, for every pixel we get a vector that goes away from the iso-surface and points at the closest edge in the normal direction.

\subsection{Conclusion}
\label{segmentation:conclusion}
%% Wrap up.

I have implemented two algorithms for segmenting pictures where the first is a simple algorithm that grows only based on the normal of the iso-surface and stops its segmentation when the iso-surface encounters pixelvalues that cross the threshold specified.

The second algorithm is more advanced and makes good use of the information from the gradient by growing in the direction of the edges and the normal.

Do to time constraints I have only tested the algorithms on grayscale pictures and also not on real medical data, but none the less I still get results that should scale to real data.



\section{Fluid / Smoke}
\label{sec:fluid}

\section{Image Vision}
\label{sec:imagevision}

\appendix

\newpage


\bibliographystyle{alpha}
\bibliography{text/levelset}


\end{document}


%%% Local Variables: 
%%% mode: latex
%%% TeX-master: t
%%% End: 
